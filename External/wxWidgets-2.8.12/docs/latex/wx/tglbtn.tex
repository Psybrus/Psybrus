%%%%%%%%%%%%%%%%%%%%%%%%%%%%%%%%%%%%%%%%%%%%%%%%%%%%%%%%%%%%%%%%%%%%%%%%%%%%%%%
%% Name:        tglbtn.tex
%% Purpose:     wxToggleButton documentation
%% Author:      Vadim Zeitlin
%% Modified by:
%% Created:     20.11.01
%% RCS-ID:      $Id: tglbtn.tex 39312 2006-05-24 09:42:25Z ABX $
%% Copyright:   (c) 2001 Vadim Zeitlin
%% License:     wxWindows license
%%%%%%%%%%%%%%%%%%%%%%%%%%%%%%%%%%%%%%%%%%%%%%%%%%%%%%%%%%%%%%%%%%%%%%%%%%%%%%%

\section{\class{wxToggleButton}}\label{wxtogglebutton}

wxToggleButton is a button that stays pressed when clicked by the user. In
other words, it is similar to \helpref{wxCheckBox}{wxcheckbox} in
functionality but looks like a \helpref{wxButton}{wxbutton}.

You can see wxToggleButton in action in the sixth page of the
\helpref{controls}{samplecontrols} sample.

\wxheading{Derived from}

\helpref{wxControl}{wxcontrol}\\
\helpref{wxWindow}{wxwindow}\\
\helpref{wxEvtHandler}{wxevthandler}\\
\helpref{wxObject}{wxobject}

\wxheading{Include files}

<wx/tglbtn.h>

\wxheading{Window styles}

There are no special styles for wxToggleButton.

See also \helpref{window styles overview}{windowstyles}.

\wxheading{Event handling}

\twocolwidtha{7cm}
\begin{twocollist}\itemsep=0pt
\twocolitem{{\bf EVT\_TOGGLEBUTTON(id, func)}}{Handles a toggle button click event.}
\end{twocollist}

\wxheading{See also}

\helpref{wxCheckBox}{wxcheckbox}, \helpref{wxButton}{wxbutton}

\latexignore{\rtfignore{\wxheading{Members}}}

\membersection{wxToggleButton::wxToggleButton}\label{wxtogglebuttonconstr}

\func{}{wxToggleButton}{\void}

Default constructor.

\func{}{wxToggleButton}{\param{wxWindow* }{parent}, \param{wxWindowID}{ id},\rtfsp
\param{const wxString\& }{label}, \param{const wxPoint\& }{pos = wxDefaultPosition},\rtfsp
\param{const wxSize\& }{size = wxDefaultSize}, \param{long}{ style = 0},\rtfsp
\param{const wxValidator\& }{val}, \param{const wxString\& }{name = ``checkBox"}}

Constructor, creating and showing a toggle button.

\wxheading{Parameters}

\docparam{parent}{Parent window. Must not be {\tt NULL}.}

\docparam{id}{Toggle button identifier. A value of $-1$ indicates a default value.}

\docparam{label}{Text to be displayed next to the toggle button.}

\docparam{pos}{Toggle button position. If the position $(-1, -1)$ is specified then a default position is chosen.}

\docparam{size}{Toggle button size. If the default size $(-1, -1)$ is specified then a default size is chosen.}

\docparam{style}{Window style. See \helpref{wxToggleButton}{wxtogglebutton}.}

\docparam{validator}{Window validator.}

\docparam{name}{Window name.}

\wxheading{See also}

\helpref{wxToggleButton::Create}{wxtogglebuttoncreate}, \helpref{wxValidator}{wxvalidator}

\membersection{wxToggleButton::\destruct{wxToggleButton}}\label{wxtogglebuttondtor}

\func{}{\destruct{wxToggleButton}}{\void}

Destructor, destroying the toggle button.

\membersection{wxToggleButton::Create}\label{wxtogglebuttoncreate}

\func{bool}{Create}{\param{wxWindow* }{parent}, \param{wxWindowID}{ id},\rtfsp
\param{const wxString\& }{label}, \param{const wxPoint\& }{pos = wxDefaultPosition},\rtfsp
\param{const wxSize\& }{size = wxDefaultSize}, \param{long}{ style = 0},\rtfsp
\param{const wxValidator\& }{val}, \param{const wxString\& }{name = ``checkBox"}}

Creates the toggle button for two-step construction. See \helpref{wxToggleButton::wxToggleButton}{wxtogglebuttonconstr}\rtfsp
for details.

\membersection{wxToggleButton::GetValue}\label{wxtogglebuttongetvalue}

\constfunc{bool}{GetValue}{\void}

Gets the state of the toggle button.

\wxheading{Return value}

Returns {\tt true} if it is pressed, {\tt false} otherwise.

\membersection{wxToggleButton::SetValue}\label{wxtogglebuttonsetvalue}

\func{void}{SetValue}{\param{const bool}{ state}}

Sets the toggle button to the given state. This does not cause a
{\tt EVT\_TOGGLEBUTTON} event to be emitted.

\wxheading{Parameters}

\docparam{state}{If {\tt true}, the button is pressed.}
